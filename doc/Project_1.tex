\documentclass[a4paper, 10pt, english]{revtex4-2} %Add reprint for two columns

\usepackage[utf8]{inputenc}
\usepackage[english]{babel}

\usepackage{physics,amssymb}  % mathematical symbols (physics imports amsmath)
\usepackage{graphicx}         % include graphics such as plots
\usepackage{xcolor}           % set colors
\usepackage{hyperref}         % automagic cross-referencing (this is GODLIKE)
\usepackage{tikz}             % draw figures manually
%\usepackage{listings}         % display code
\usepackage{subfigure}        % imports a lot of cool and useful figure commands
\usepackage{array}
\usepackage{microtype}
\usepackage[export]{adjustbox}

\tolerance=1
\emergencystretch=\maxdimen
\hyphenpenalty=10000
\hbadness=10000

% defines the color of hyperref objects
% Blending two colors:  blue!80!black  =  80% blue and 20% black
\hypersetup{ % this is just my personal choice, feel free to change things
    colorlinks,
    linkcolor={red!50!black},
    citecolor={blue!50!black},
    urlcolor={blue!80!black}}

\newcommand{\txt}[1]{\text{#1}}
\newcommand{\parder}[2]{\frac{\partial #1}{\partial #2}}
\newcommand{\dder}[2]{\frac{\partial^2 #1}{\partial #2^2}}

\begin{document}
\vspace*{1.5cm}
\title{\LARGE Computational physics: Project 1}
\author{Jon Aleksander Prøitz}
\date{\today}
\noaffiliation
\maketitle

\section*{\large Section 1}
    The local energy is given by
    \begin{equation}
            E_L 
        =   \frac{1}{\Psi_T}H\Psi_T
    \end{equation}
    with
    \begin{equation}
            H
        =   \sum_i^N \left(\frac{-\hbar^2}{2m} \nabla_i^2 + V_\txt{ext}\right) + \sum_{i<j}^N V_\txt{int}
    \end{equation}
    \begin{equation}
            V_{ext}(\mathbf{r}) 
        =   \Bigg\{
        \begin{array}{ll}
            \frac{1}{2}m\omega_{ho}^2r^2 & (S)\\
        \strut
            \frac{1}{2}m[\omega_{ho}^2(x^2+y^2) + \omega_z^2z^2] & (E)
        \label{trap_eqn}
        \end{array}
    \end{equation}
    \begin{equation}
        V_{int}(|\mathbf{r}_i-\mathbf{r}_j|) =  \Bigg\{
        \begin{array}{ll}
            \infty & {|\mathbf{r}_i-\mathbf{r}_j|} \leq {a}\\
            0 & {|\mathbf{r}_i-\mathbf{r}_j|} > {a}
        \end{array}
    \end{equation}
    \begin{equation}
        \Psi_T(\mathbf{r})=\Psi_T(\mathbf{r}_1, \mathbf{r}_2, \dots \mathbf{r}_N,\alpha,\beta)
        =\left[
           \prod_i g(\alpha,\beta,\mathbf{r}_i)
        \right]
        \left[
           \prod_{j<k}f(a,|\mathbf{r}_j-\mathbf{r}_k|)
        \right],
        \label{eq:trialwf}
    \end{equation}
    for $a=0$, $\beta = 1$, note that in 1D
    \begin{equation}
            \frac{1}{\Psi_T}\dder{\Psi_T}{x}
        =   -2\alpha\frac{1}{\Psi_T}\parder{\Psi_T}{x}\left(x\Psi_T\right)
        =   2\alpha{\left(2\alpha x^2 - 1\right)}
    \end{equation}
    \begin{equation}
            \frac{1}{\Psi_T} \sum_{i=1}^d \dder{\Psi_T}{x_i}
        =   2\alpha\left(2\alpha\sum_{i=1}x_i^2 - d\right)
    \end{equation}



\end{document}
